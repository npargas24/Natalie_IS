%!TEX root = ../main.tex
\chapter{Conclusions and Future Work}
\section{Reverse Engineering Outcome}
This project has undertaken a comprehensive investigation of reverse engineering practices as applied to complex Bluetooth‑enabled systems. By examining both theoretical foundations and practical implementations, I have illuminated the interplay between static code analysis, dynamic instrumentation, and wireless packet inspection. The objective was twofold: to understand the internal architecture of a commercial Android application that communicates with a Bluetooth peripheral and to demonstrate the feasibility of intercepting and manipulating its behavior in real time.

First, static analysis using JADX provided a crucial starting point for mapping the application’s data flow and identifying key functions. Through systematic decompilation and code review, I traced how user inputs and device commands are translated into Bluetooth payloads. This phase revealed the structure of the proprietary protocol and established a blueprint for subsequent dynamic testing. However, static techniques alone cannot capture runtime behaviors such as input validation, encryption routines, or environment‑specific logic.

To address these limitations, I employed Frida for dynamic instrumentation. By injecting hooks into the application’s Java methods, I was able to observe live function calls, inspect variable values, and modify parameters before transmission. Frida’s versatility enabled close inspection of runtime behavior and the execution of targeted manipulations, confirming that the application’s control logic could be influenced without source‑level modifications. This dynamic approach complemented the insights gained from static decompilation and provided definitive evidence of our capacity to alter an application’s operation.

Complementing code‑centric methods, Bluetooth packet sniffing yielded an unambiguous record of over‑the‑air exchanges between the mobile application and the peripheral device. Using a hardware sniffer and analysis software, I captured raw packets that corroborated the structure inferred from code review and Frida instrumentation. These traces ensured that our reverse‑engineering conclusions were grounded in empirical measurements rather than extrapolation. The integration of packet‑level data with code‑level insights formed a robust, end‑to‑end understanding of the entire communication pipeline.

Throughout this work, an iterative methodology proved essential. Each new discovery prompted reevaluation of prior assumptions and guided subsequent experiments. Establishing a reliable environment—including rooted devices and virtual machines—facilitated rapid prototyping and reproducibility. Equally important was meticulous documentation of all procedures, findings, and hypotheses, which preserved critical knowledge and supported collaborative scrutiny.

In sum, this project demonstrates that commercial‑grade 
Bluetooth applications can be effectively reverse engineered through a judicious combination of static analysis, dynamic instrumentation, and packet monitoring. The techniques and workflows developed here provide a foundation for further research into more advanced protocols and emerging hardware platforms. With refined automation tools and expanded coverage to additional communication layers, future work can continue to push the boundaries of applied reverse engineering within the security research community.

\section{Future Work}
The current study has demonstrated that manipulating the dispense process of the device is not only possible but can serve as a robust foundation for achieving comprehensive command over its functions. Building upon these initial findings, the next phase of research could refine communication protocols, reverse‑engineer proprietary algorithms, develop a standalone application, and analyze firmware to uncover hidden operational logic.

First, building on the initial packet capture efforts conducted during this study, the research could return to in‑depth Bluetooth packet analysis to further elucidate the device’s communication protocol. This effort might involve capturing and parsing additional raw GATT traffic to confirm previously observed services, characteristics, and command sequences. Using tools such as Wireshark or specialized BLE sniffers, the study could refine documentation of packet structures, attribute handles, and payload formats. 
Second, additional efforts could concentrate on isolating and interpreting the device’s color-related commands within the captured Bluetooth packets. This would involve identifying specific GATT characteristics or payload patterns that correspond to shade adjustments, enabling direct control over pigment ratios without requiring extensive algorithmic reconstruction.

Third, with both communication protocols and color models established, the research could focus on developing a standalone mobile application independent of the vendor’s software. This application could include detailed documentation of the Bluetooth services and characteristics, such as UUIDs, properties, and data formats, required to interface with the device. The user interface could then offer intuitive workflows for shade selection, custom color creation, and real‑time monitoring.
Finally, to reveal any embedded processing routines not exposed through the GATT interface, the study could perform firmware extraction and static analysis. Through hardware teardown and debugging techniques, the research could retrieve the device’s firmware image. 

Reverse‑engineering tools could then be employed to identify internal state machines, calibration sequences, and safety checks. Mapping these routines to observed device behaviors could close gaps in the external command model and inform enhancements to custom control.

In conclusion, the proposed future work lays out a clear path for turning early experiments into a fully independent, user-controlled system for accurate color dispensing. By improving communication with the device, learning how its internal processes work, creating a custom app, and examining its internal software, this research could move entirely away from relying on the original manufacturer’s app and open up new possibilities for personalized cosmetic production.
