%%%%%%%%%%%%%%%%%%%%%%%%%%%%%%%%%%%%%%%%%%%%%%%%%%%%%%%
%
%                                                       Example IS Template
%
% \documentclass{woosterthesis} must be at the beginning of every IS. Options are the same as
% for the report class with some additional options: abstractonly, acs, alltt, apa, blacklinks, chicago,
% code, colophon, dropcaps, euler, foreignlanguage, gauss, index, kaukecopyright, lshortwooster, maple, mla, palatino, picins,tikz,
% verbatim, wblack, and woostercopyright. The kaukecopyright option will put the arch symbol with the word mark on the
% copyright page. The woosterthesis class is based on the report class. One thing to note is that
% the ``%'' symbol comments out all characters that follow it on the line.
%
%%%%%%%%%%%%%%%%%%%%%%%%%%%%%%%%%%%%%%%%%%%%%%%%%%%%%%%

%%%%%%%%%%%%%%%%%%%%%%%%%%%%%%%%%%%%%%%%%%%%%%%%%%%%%%%
%
% Checked on 8/26/22 and compiles with no fatal errors. Users must have the latest version of the TeXLive software and
% have installed all available packages from CTAN to ensure this thesis class compiles with no fatal errors. Also, you must
% run pdfLaTex, Biber, MakeIndex, padfLaTeX, pdfLaTeX to get all the numbering and references resolved. This is the
% first year the template uses Biber for references.
%
%%%%%%%%%%%%%%%%%%%%%%%%%%%%%%%%%%%%%%%%%%%%%%%%%%%%%%%

%%%%%%%%%%%%%%%%%%%%%%%%%%%%%%%%%%%%%%%%%%%%%%%%%%%%%%%
% use this declaration for a draft  version of your IS
%\documentclass[10pt,palatino,code,picins,tikz,kaukecopyright,openright,lshortwooster,dropcaps,verbatim,index,euler]{woosterthesis}
% Note that you can specify the acs option to use the American Chemical Society citation format, apa option to use the American
% Psychological Association citation format, chicago option to use the Chicago citation format, mla option to use the Modern Language
% Association citation format, wblack option for a grayscale Wooster "W" on the cover, and scottie option for a grayscale Scottie Mascot
% on the cover. See exampleis_manual.pdf for an explanation of all the options.
%
%%%%%%%%%%%%%%%%%%%%%%%%%%%%%%%%%%%%%%%%%%%%%%%%%%%%%%%
%
% use this declaration for the print version of your IS
%\documentclass[12pt,code,palatino,picins,blacklinks,kaukecopyright,openright]{woosterthesis} % probably what most students would use
%
%%%%%%%%%%%%%%%%%%%%%%%%%%%%%%%%%%%%%%%%%%%%%%%%%%%%%%%
%
% use this declaration for the PDF version of your IS
\documentclass[12pt,code,palatino,picins,kaukecopyright,openright]{woosterthesis}
%
%%%%%%%%%%%%%%%%%%%%%%%%%%%%%%%%%%%%%%%%%%%%%%%%%%%%%%%

%%%%%%%%%%%%%%%%%%%%%%%%%%%%%%%%%%%%%%%%%%%%%%%%%%%%%%%
%
%                                                       Load Packages
%
%   To load packages in addition to the ones that are loaded by default, please place your
%   usepackage commands in the packages.tex file in the styles folder.
%
%%%%%%%%%%%%%%%%%%%%%%%%%%%%%%%%%%%%%%%%%%%%%%%%%%%%%%%

\input{styles/packages}

%%%%%%%%%%%%%%%%%%%%%%%%%%%%%%%%%%%%%%%%%%%%%%%%%%%%%%%
%
%                                                       Load Personal commands
%                                                                    
%  There will be certain commands that you use frequently in the thesis. You can give these
%  commands new names which are easier for you to remember. You can also combine several
%  commands into a new command of your own. See The LaTeX Companion or Guide to LaTeX
%  for examples on defining your own commands. These are commands that I defined to cut
%  down on typing. You can enter your commands in the personal.tex file in the styles folder.
%
%%%%%%%%%%%%%%%%%%%%%%%%%%%%%%%%%%%%%%%%%%%%%%%%%%%%%%%

\input{styles/personal}

%%%%%%%%%%%%%%%%%%%%%%%%%%%%%%%%%%%%%%%%%%%%%%%%%%%%%%%
%
%                                                       Load Theorem formatting information
%
%  If you need to define an new theorem style or want to see what theorem like environments 
%  are available please look at the theorems.tex file in the styles folder.
%
%%%%%%%%%%%%%%%%%%%%%%%%%%%%%%%%%%%%%%%%%%%%%%%%%%%%%%%

\input{styles/theorems}

\setcounter{secnumdepth}{5}% controls the numbering of sections
\setcounter{tocdepth}{6}% controls the number of levels in the Contents

%%%%%%%%%%%%%%%%%%%%%%%%%%%%%%%%%%%%%%%%%%%%%%%%%%%%%%%
%
%  This is where one enters their bibilography file name.
%
%%%%%%%%%%%%%%%%%%%%%%%%%%%%%%%%%%%%%%%%%%%%%%%%%%%%%%%

\addbibresource{references.bib}

%%%%%%%%%%%%%%%%%%%%%%%%%%%%%%%%%%%%%%%%%%%%%%%%%%%%%%%
%
%  This is where one enters the information about the thesis.
%
%%%%%%%%%%%%%%%%%%%%%%%%%%%%%%%%%%%%%%%%%%%%%%%%%%%%%%%

\title{Reverse Engineering a Bluetooth Application: Discovering the Secrets of a Makeup Printing Device}
\thesistype{Independent Study Thesis} % you should make this Independent Study Thesis
\author{Natalie Pargas}
%\presentdegrees{Ph.D.} % you should comment this line
\degreetoobtain{Bachelor of Arts in Computer Science}
\presentschool{The College of Wooster}
\academicprogram{Department of Mathematical \& Computational Sciences}
\gradyear{2025}
\advisor{Professor Drew Guarnera (Computer Science)}
%\secondadvisor{Professor Heather Guarnera (Department of Mathematical \& Computational Sciences)}
%\reader{Reader}
\copyrighted   
%\copyrightdate{}                  
\makeindex % comment this line if you do not have an index

%%%%%%%%%%%%%%%%%%%%%%%%%%%%%%%%%%%%%%%%%%%%%%%%%%%%%%%
%
%  This is where the commands for the document begin. All \LaTeX{} documents must have a
%  \begin{document} text .... \end{document} structure.
%
%%%%%%%%%%%%%%%%%%%%%%%%%%%%%%%%%%%%%%%%%%%%%%%%%%%%%%%

\begin{document}

%%%%%%%%%%%%%%%%%%%%%%%%%%%%%%%%%%%%%%%%%%%%%%%%%%%%%%%
%
%  The front matter includes acknowledgments, dedications, vitas, list of tables, list of figures,
%  copyright, abstract, title page, and contents.
%
%%%%%%%%%%%%%%%%%%%%%%%%%%%%%%%%%%%%%%%%%%%%%%%%%%%%%%%

\frontmatter
\maketitle
\ClearShipoutPicture
\clearpage\thispagestyle{empty}\null\clearpage
\disscopyright 

%%%%%%%%%%%%%%%%%%%%%%%%%%%%%%%%%%%%%%%%%%%%%%%%%%%%%%%
%                                                                                       
%                                                       Abstract						
%                                                                                       
%%%%%%%%%%%%%%%%%%%%%%%%%%%%%%%%%%%%%%%%%%%%%%%%%%%%%%%

\begin{abstract}
The abundance and ubiquity of small, bespoke, Internet of Things (IoT) devices has widened a growing knowledge gap between what is understood about them and their actual capabilities. Thousands of new devices are released each year, where each variety can see millions of units produced and sold. Closing or narrowing this gap requires a coordinated analysis effort, necessitating bolstered numbers of individuals with the knowledge to address this issue. However, without access to source code, build system, and detailed software/hardware documentation, this cannot be done in the traditional analysis or forward engineering paradigm. In contrast, reverse engineering is the iterative process and methodology by which hardware and software can be understood with little or no access to the information and resources the developers used to make a given device. This paper serves as a guide on how to approach and successfully reverse engineer a device to the point of being able to augment its' functionality.
\end{abstract}

%%%%%%%%%%%%%%%%%%%%%%%%%%%%%%%%%%%%%%%%%%%%%%%%%%%%%%%
%                                                                                       
%                                                       Dedications					
%                                                                                       
%%%%%%%%%%%%%%%%%%%%%%%%%%%%%%%%%%%%%%%%%%%%%%%%%%%%%%%

\dedication{This work is dedicated to my mother.}


%%%%%%%%%%%%%%%%%%%%%%%%%%%%%%%%%%%%%%%%%%%%%%%%%%%%%%%
%                                                                                       
%                                                       Acknowledgments					
%                                                                                       
%%%%%%%%%%%%%%%%%%%%%%%%%%%%%%%%%%%%%%%%%%%%%%%%%%%%%%%

\begin{acknowl}  
First, I would like to acknowledge my advisor, Professor Drew Guarnera. His guidance and encouragement during our advising sessions were instrumental in helping me see this project through. Despite the many challenges that arose along the way, his support consistently lit a path forward.

I also owe deep gratitude to my second advisor, Professor Heather Guarnera. I would not have even begun this project without her. She has championed me in every academic aspect, and it was her belief in me that allowed me to believe that this project was possible in the first place. I will forever be grateful for her counsel and motivation. She has gone far beyond what would be considered exceptional for an advisor, and I am so fortunate to have had her in my corner.

Next, I would like to thank Dean Bowen for her unwavering support with any college-related matters that threatened to impede my progress outside the classroom. She has been a reliable source of information and encouragement throughout this process. I have yet to meet someone as efficient and responsive when it comes to answering emails and finding solutions.

Finally, I would like to extend my gratitude to the entire Department of Mathematical and Computational Sciences. This has always been a department that truly cares about student learning and is willing to meet students where they are. It is through the hard work and dedication of my professors that I have been able to grow academically and, ultimately, complete this project.

\end{acknowl}

%%%%%%%%%%%%%%%%%%%%%%%%%%%%%%%%%%%%%%%%%%%%%%%%%%%%%%%
%                                                                                       
%                                                       Vita					
%                                                                                       
%%%%%%%%%%%%%%%%%%%%%%%%%%%%%%%%%%%%%%%%%%%%%%%%%%%%%%%

%\begin{vita} 
% You talk about yourself and how you got to where you are now. There is a structured form for the Vita that can be used if you want, but I don't encourage it.

%%%%%%%%%%%%%%%%%%%%%%%%%%%%%%%%%%%%%%%%%%%%%%%%%%%%%%%
%
%  The list below is for a thesis that requires a more structured Vita such as a masters or Ph.D.
%
%%%%%%%%%%%%%%%%%%%%%%%%%%%%%%%%%%%%%%%%%%%%%%%%%%%%%%%

%\begin{datelist}
%\item[August 11, 1990]Chosen to present an undergraduate paper at the 75th meeting of the MAA, Columbus, Ohio
%\item[August 1990--August 1991]President Wooster Student Chapter of the MAA, The College of Wooster, Wooster, Ohio
%\item[August 1991--May 1992]Secretary Wooster Student Chapter of the MAA, The College of Wooster, Wooster, Ohio
%\item[1992]\emph{Phi Beta Kappa} (on junior standing), The College of Wooster, Wooster, Ohio
%\item[1992]Elizabeth Sidwell Wagner Prize in Mathematics, The College of Wooster
%\item[1992]William H. Wilson Prize in Mathematics, The College of Wooster
%\item[May 11, 1992]B.A., Mathematics, The College of Wooster
%\item[1997]Finalist for Graduate Teaching Award, The Ohio State University, Columbus, Ohio
%\item[June 21-25, 1998]Participant in the AMS-IMS-SIAM Summer Research Conferences: q-Series, Combinatorics, and Computer Algebra, Mt. Holyoke, Massachusetts
%\item[October 1998--October 1999]Graduate student representative to The Ohio State University Department of Mathematics Graduate Studies Committee, Columbus, Ohio
%\item[January 1999]q-series seminar address, The Ohio State University, Columbus, Ohio
%\item[2000]Finalist for Departmental Teaching Award, The Ohio State University, Columbus, Ohio
%\item[2000]Nominated for Graduate Teaching Award, The Ohio State University, Columbus, Ohio
%\item[April 2000]Invited colloquium talk at The College of Wooster, Wooster, Ohio
%\item[1992-- present]Graduate Teaching and Research Associate, The Ohio State University
%\end{datelist}
%
%%%This is for any publications you might have.%%%%%

%\begin{publist}  
%\pubitem{\quad}
%\pubitem{\quad}
%\end{publist} 

%\begin{fieldsstudy} 
    %\majorfield{Computer Science}
    %\specialization{Area of IS research}
    %\begin{studieslist}
   %\studyitem{Abstract Algebra}{Hampton}
   %\end{studieslist}
  %\end{fieldsstudy}
%\end{vita}

%%%%%%%%%%%%%%%%%%%%%%%%%%%%%%%%%%%%%%%%%%%%%%%%%%%%%%%
%
%  We now create the contents page and if necessary the list of figures and list of tables.
%
%%%%%%%%%%%%%%%%%%%%%%%%%%%%%%%%%%%%%%%%%%%%%%%%%%%%%%%


\cleardoublepage
\phantomsection
\addcontentsline{toc}{chapter}{Contents}

\tableofcontents
\listoffigures %Use if you have a list of figures.
\listoftables%Use if you have a list of tables.
\lstlistoflistings% Use if you are using the code option

%%%%%%%%%%%%%%%%%%%%%%%%%%%%%%%%%%%%%%%%%%%%%%%%%%%%%%%


%%%%%%%%%%%%%%%%%%%%%%%%%%%%%%%%%%%%%%%%%%%%%%%%%%%%%%%
\mainmatter

%%%%%%%%%%%%%%%%%%%%%%%%%%%%%%%%%%%%%%%%%%%%%%%%%%%%%%%
%
%                                                       Thesis Chapters
%
% This is where the main text of the thesis goes. I have written this template assuming that
% each chapter is a separate file. You do not have to do this but it makes things easier to find
% for editing. You can use the sample chapters to help you figure out how to type things into
% your thesis. To include a chapter just use the \include{chaptername} command. Chapters are
% included in the order listed.
%
%%%%%%%%%%%%%%%%%%%%%%%%%%%%%%%%%%%%%%%%%%%%%%%%%%%%%%%

%!TEX root = ../main.tex\
\chapter{Introduction}\label{intro}
\section{What is Reverse Engineering?}

Reverse engineering is the process of disassembling, deconstructing, or analyzing a system, device, or piece of software in order to gain a deep understanding of how it functions. This concept has existed long before its modern applications in technology. For example, the dissections many of us performed in high school biology classes were, in essence, an early form of reverse engineering. Those dissections were carried out to break down something as complex and mystifying as life into understandable components and functions, revealing how various biological systems worked together. Similarly, one of the most effective ways to fully comprehend how a technological system operates is to open it up and inspect it piece by piece.

Reverse engineering software can be accomplished through a variety of techniques. Broadly speaking, these methods fall into two categories: static analysis and dynamic analysis. Static analysis involves studying a snapshot of the code in a single, unchanging state without executing it. This provides insight into how the software is structured, its logic, and potential areas of interest. In contrast, dynamic analysis takes place while the program is actively running, allowing an analyst to observe how it behaves in real time, including how it handles data and responds to inputs. Typically, a reverse engineering process might begin by passing a program’s executable to a disassembler, which translates machine code into assembly-level instructions. From there, a programmer might use system monitoring tools to gather information provided by the operating system about how the program interacts with its environment. Additionally, debuggers allow reversers to observe the CPU’s internal operations, stepping through the disassembled code one instruction at a time to scrutinize the logic behind each operation. Using these various tools, a programmer must rely heavily on intuition, experience, and problem-solving skills to navigate the complex and often opaque process of reverse engineering.

There are numerous reasons someone might choose to reverse engineer software. The only time the inner workings of software are openly accessible to anyone is when it is released as open source. Open source software consists of freely available source code that anyone can read, modify, or distribute. However, developers frequently choose not to make their source code public for a range of reasons, including legal and licensing restrictions, protection of proprietary technologies, or personal business strategies. The opacity of proprietary software often leaves users, researchers, or competing developers with no choice but to reverse engineer if they wish to understand or interact with it.

Some of the most common motivations for reverse engineering software include performing malware analysis, improving one’s programming skills, recovering lost source code, and enabling interoperability between different systems or applications. My own first introduction to the world of reverse engineering came through a video in which the creator was reverse engineering Apple’s FaceTime for Mac. His goal was to reinstate closed captioning functionality for his hearing-impaired mother—a striking example of how reverse engineering can serve highly personal and practical purposes. The potential applications of reverse engineering are as diverse and far-reaching as the software programs that exist today.

Most software developers are familiar with reverse engineering primarily through its use in malware analysis. A classic example of malware is a Trojan virus, in which malicious developers conceal harmful code inside programs that appear harmless. Reverse engineering such programs allows security researchers to break them apart and reveal the hidden malware lurking within. Additionally, modern software development often involves the use of external libraries. In large projects, keeping track of all dependencies and the interactions between different modules can be a formidable challenge. Hackers can exploit this complexity by hiding malicious entry points in these libraries to gain unauthorized access to sensitive information. In many cases, reverse engineering remains the only foolproof method for uncovering the true operations of a program and exposing any hidden threats or vulnerabilities embedded deep within the code.

Another compelling reason to reverse engineer software is to enhance one’s own technical skills. The process requires extensive knowledge of both low-level assembly language and high-level programming concepts. Reverse engineering someone else’s code is no simple feat. It forces the reverser to learn deeply about various methods of structuring and implementing software, as well as the diverse strategies developers use to solve problems. Through this process, reverse engineers often become far more proficient programmers, gaining insight into advanced coding techniques and architectural design choices that might not be visible in standard learning materials. This project itself will serve as an endeavor in learning not only forward engineering but also the intricate art and science of reverse engineering.

Another reason people may reverse engineer software is to recover lost source code. Though this is far from easy, it is sometimes possible to reconstruct high-level source code from compiled executables. Many developers have experienced the frustration and panic that comes with losing source code due to accidental deletion, hardware failures, or lack of proper backups. In such scenarios, reverse engineering can offer a potential solution, providing a lifeline to recover critical work that might otherwise be lost forever.

Interoperability represents yet another crucial motivation for reverse engineering. In many cases, achieving compatibility between different programs, systems, or devices is impossible without reverse engineering. When working with external libraries or operating systems that provide documentation without accompanying source code, developers frequently encounter undocumented behaviors or missing details. While a programmer could attempt to resolve these issues through trial and error or by contacting the vendor, reverse engineering provides a definitive and often faster way to understand how to integrate different systems effectively. It serves as a bridge between incompatible software components, enabling seamless communication and cooperation in ways that might otherwise remain impossible.

This project serves as a case study in using reverse engineering to achieve interoperability with a proprietary Bluetooth device.

%!TEX root = ../main.tex
\chapter{Reverse Engineering Methods, Tools, and examples}
Because reverse engineering serves a wide variety of purposes, there is an equally broad range of methods and strategies employed to achieve those goals. In \textit{Reversing: Secrets of Reverse Engineering}, these methods are generally categorized into two primary types. The first type focuses on large-scale observation of a program and is referred to as system-level reversing \cite{Reversing}. System-level reversing seeks to determine the overarching structure of a program and identify potential areas of interest within it. Once a general understanding of the program’s layout has been established, the analysis can proceed to the second type: code-level reversing. Code-level reversing delves deeper into the specifics of a program’s inner workings, offering detailed insights into particular segments of code and uncovering how algorithms and functionalities are implemented.

System-level reversing is achieved by running an array of specialized tools on a program and observing the services provided by the operating system. These observations gather information on how the program interacts with system resources, tracks input and output operations, and reveals characteristics of the program’s executable files. Much of this data is supplied by the operating system, as any portion of a program that interacts with components outside itself must necessarily communicate through the OS. As a result, effective system-level reversing demands a strong familiarity with the operating system’s internal workings, making it invaluable for reverse engineers to develop deep expertise in the OS environment they are targeting.

Code-level reversing, by contrast, is a far more complex and meticulous process. It involves extracting algorithms, logic, and design concepts directly from a program’s binary code. This task requires a robust background in reverse engineering techniques, software development practices, CPU architecture, and operating system internals. Even when working with readable, well-documented source code, modern software systems can be enormously complex. For instance, the program examined in this project comprises 21 frameworks, contributions from at least three separate companies, and potentially tens or even hundreds of thousands of lines of code. Moreover, when high-level code is compiled into binary instructions, a single line of code can transform into ten or more lines of low-level machine instructions. Reversing production-level software from binary back to source code is thus an almost incomprehensibly difficult endeavor. Thankfully, multiple methodologies and a variety of specialized tools have been developed to aid in effective code-level reversing, making this seemingly impossible task more approachable for skilled practitioners.

\subsection{Static Analysis}

Static analysis is the practice of analyzing code without executing it. This approach allows reverse engineers to study the structural and logical components of a program in a controlled environment, free from the risks associated with running potentially malicious code. Static analysis can also serve as a valuable precursor to dynamic analysis, helping to identify key areas of interest for more in-depth examination later.

Static analysis tools operate at both the source code level, when code has not yet been compiled, and at the machine language level, analyzing binary or assembly-level code. Reverse engineers frequently use static analysis tools to disassemble machine code into assembly instructions, enabling them to examine and interpret how different sections of code relate to one another. While retrieving disassembled code is one of the most fundamental functions of static analysis tools, many modern tools offer a host of additional features, such as graphical representations of control flow, cross-referencing functions, and identifying potential vulnerabilities. These tools are often designed to provide reverse engineers with a starting point, helping them navigate the often bewildering world of assembly-level functions and binary logic.

\subsubsection{Disassembler}
One of the most key static analysis tools for a reverse engineer is a disassembler. 
Disassemblers are applications that take in a program’s executable file and generate a file with the assembly code.
This isn’t too difficult because assembly is just a text translation of the machine readable binary code. 
Unfortunately, that makes it a lot less clear than higher level programming languages.  
Disassembly is also unique to a person’s processor, but there are disassemblers that can support more than one CPU architecture.
Some examples of disassemblers are IDA, Binary Ninja, Hopper, Ollydbug, Radare2, and Ghidra.

\subsubsection{Decompiler}
Decompilers are similar but more complex than disassemblers.
Instead of just producing the assembly language code, decompilers attempt to produce high-level readable code. 
They attempt to produce something that looks as close to the source code as possible by reversing the compilation process~\cite{Reversing}.
The front end of the decompiler decodes low-level assembly instructions and translates it into an intermediate representation specific to the decompiler. 
The intermediate representation is then iterated on to remove as many extraneous details as possible while preserving the important details. 
Lastly the back end takes the polished up intermediate representation and translates it again into a high level language.

While this may sound like a simple way to retrieve the source code, it is highly unlikely the high level language returned will be something that actually matches. 
Anyone who has ever run text through a translation website multiple times knows that each iteration causes a loss in fidelity and oftentimes the end result will be gibberish. 
Decompilers go a step further and have no immediate knowledge of things like function or variable names and structures. 
This does not make them useless, though. For example, a text translator will usually return your original result when given a simple phrase. 
Similarly a decompiler will most likely be able to pick up on more straightforward functions such as adding a + b. 
Decompilers offer hints and snippets into what the source code may look like which is valuable information for a reverse engineer. 
Most of the examples for disassemblers listed also count as decompilers.

\subsubsection{Dumping Tools}
Executable dumping is usually the first step in reverse engineering a program. 
Executable dumping helps inform a reverse engineer what a program does, how it interacts with external elements, and general knowledge of how a file is structured. 
Using dumping tools, a reverse engineer will start by figuring out what type of file they’re dealing with. 
They will then start to unpack the format. 
Extracting strings will immediately offer insight into what language the source code is written in, what library modules it uses, what kinds of data it is dealing with, and what the goals of program functions may be. 
Other useful information we can retrieve is the layout of the file in memory and what the general logical structure of the file is. Once we know the type of file we also know which tools we can use. 
For example, some debuggers only work with certain high and low level languages. 
Some popular executable dumping programs are ones already built into your machine like DUMPBIN for Windows or otool for Mac. 

Otool is identifies a file's Mach header. 
On systems with a Mach kernel like macOS and iOS, the executable binaries they use will have a Mach header. 
All headers have a magic number to identify them. Thin binaries only have that number while fat binaries with fat headers will have locations of the other executable's headers in them.

\subsection{Dynamic Analysis}

Dynamic analysis occurs when a program—or specific sections of it—is executed during the analysis process. This approach provides unique insights into how software behaves in real time, revealing execution paths, input/output relationships, and interactions with external systems. Because running an unknown program can produce unpredictable and potentially harmful results, it is safest to conduct dynamic analysis within a secure, isolated, or sandboxed environment. Virtual machines are commonly used for this purpose, as they are easy to set up, control, and reset between tests.

A wide variety of tools contribute to dynamic analysis. Any tool that executes code and monitors its behavior falls under this category. Some dynamic analysis tools run entire programs to observe overall functionality, while others focus on specific sections, functions, or even single instructions. Additionally, tools that monitor a program’s external environment during execution—for example, capturing network traffic, tracking file system changes, or observing registry modifications—are also considered part of dynamic analysis. Together, these tools give reverse engineers a powerful way to understand not just how a program is built, but how it behaves in practice.

\subsubsection{Debugger}
Debuggers are a tool usually used for developers to locate and work through errors in their programs, but they’re also vital tools in reverse engineering. 
Many debuggers can work through assembly language. 
While assembly language may be difficult for a human to parse through, it is the exact same logic that gets broken down and sent to a computer meaning that the computer can show what each assembly instruction is intended to do. 
Most debuggers software engineers use were actually designed from the ground up with the purpose of stepping through assembly code. 
The debugger can show the state of CPU registers along with a memory dump that shows what’s in the stack. 
Debuggers usually contain disassemblers which as discussed is a vital reverse engineering tool. 
Many debuggers also contain both software and hardware breakpoints. 
Software breakpoints are instructions added into the code during runtime to pause and hand control over to the debugger. 
Hardware breakpoints are a CPU feature which allow the processor to pause execution when a certain memory address is reached and hand control over to the debugger. 
A reverse engineer can use insert breakpoints at data structures of interest and use the debugger to reveal what they are. 
Debuggers also offer a clear view of registers and memory to aid in a reverse engineer's ability to process low level information.

\paragraph{\underline{User Mode Debuggers}}
Most debuggers that a programmer will use are user mode debuggers.
They run on a system like any other application then seize control of the target program to debug ~\cite{Reversing}.
An advantage to them is that they are easier to set up and navigate than their kernel-mode counterparts.
Usually it’s fine to stay limited to user mode viewing of an application. 
It only creates troublesome limitations when the target application has kernel mode components such as device drivers. 
User mode debuggers are also not always sufficient when trying to debug a program before it reaches the main entry point. 
These kinds of programs are usually ones that have a lot of statically linked libraries in the executable. 
The final and most likely to be problematic issue is that user-mode debuggers can only view a single process in a program. 
This can cause issues when the target application has processes that interact in unknown ways. 
The user may not know which process to zero in on that has the code of interest.

\paragraph{\underline{Kernel Mode Debuggers}}
Kernel mode debuggers are different from user-mode debuggers because they capture the entire system and not just a single process. 
Instead of running atop the operating system, kernel mode debuggers sit alongside the system kernel and stay ready to capture the entire system’s stats. 
Kernel mode debuggers can be more helpful to reverse engineers because they offer more clues as to what’s going on with the system. 
A key tool for kernel-mode debuggers is that they allow the placement of low level code breakpoints. 
As a reverse engineer working with assembly code, being able to test different sets of assembly instructions that may be the code section a reverser is looking for is incredibly useful.

For example, picture a scenario where a reverse engineer is looking for the API responsible for handling moving windows? 
That will need to be managed by a windows manager in the kernel. 
The complexity is introduced when trying to identify which API moves a particular window. 
Since there are multiple APIs that can be used to move a window, pinpointing the one you are trying to focus on is a difficult task. 
This is where kernel mode debuggers come in handy. 
Low level breakpoints in the operating system responsible for shifting windows around will help you identify which API is getting called when a window is moved~\cite{Reversing}. 
From there the reverse engineer will be able to hone in on that API.

Unfortunately, kernel mode debuggers come with their own drawbacks. 
Setting them up can be challenging because they require access to the full system. 
They need to suspend the entire system while running so they can go line by line, which means the system can no longer have multiple threads going~\cite{Reversing}. 
They will also usually need to be set up inside a virtual machine which will be discussed in a later section~\cite{PracticalRE}. 
These are reasons why a reverser should exhaust their other options before jumping into using a kernel mode debugger. 
Still, they are powerful tools when the scenario calls for it.

\subsubsection{System Monitor}
System monitoring can be a crucial part of the reverse engineering process. 
In some cases, a reverse engineer can figure out what they need through system monitoring tools without ever looking at the decompiled code. 
System monitoring tools can capture what happens between the code and hardware using the intermediate channels of input/output~\cite{Reversing}. 
For local system monitoring, tools monitor things such as file operations (creating, deleting, moving, etc). 
For programs that communicate over networks, system monitoring might look like recording all TCP/UDP network traffic. 
System monitoring tools are commonly used in virtual machines.


\subsubsection{Virtual Machine}
Many programmers will encounter virtual machines (VM) at some point during their career. 
It’s not uncommon for an application to interface exclusively with one type of operating system. 
One reason is because writing applications to work with different operating systems adds time and complexity that not all programmers can afford.
Choosing to write in a high level language may make code more portable because there’s more verbose built-in libraries or interpreters that deal with the specifics of the systems without the need for any programmer intervention.
But high level languages often trade performance for convenience which is not a trade that a developer working with large amounts of data can necessarily afford to take.
What happens when a developer has an application that runs on Windows while they have a Mac? 
This is where a virtual machine enters into the mix. Virtual machines are safe sandboxed environments that work as miniature computers with their own operating systems within a computer. 
They are constructed with an interpreter and bytecode. 
During compile time, select code snippets are compiled specific for the VM target architecture and then inserted into the program along with the interpreter. 
During run-time, the interpreter begins executing the bytecode. 
Virtual machines are costly to implement because they are so expansive which is why only the necessary code snippets get rendered~\cite{MasteringRE}.
Because virtual machines work in their own isolated environment, they offer powerful protections against unknown or potentially malicious software. 
This makes them a useful tool for reverse engineers who deal with software that they do not know the contents of. 
The level of control over virtual machines also makes them a useful tool because there is ultimate knowledge of system conditions.


\section{Protocol Analysis}

Communication protocols form the essential foundation for structured exchanges of information between different systems and components. Many applications, especially those that involve interoperability between different platforms or devices, depend on well-defined formats for sending and receiving messages. However, in cases where official protocol specifications are unavailable, incomplete, or proprietary, protocol reverse engineering can be a crucial tool for discovering the format, flow, and logic behind these communications.

Protocols are extensively used in telecommunications, networking, and distributed systems because they establish consistent rules for exchanging data between different entities. Protocols may be published as open standards, freely accessible for anyone to implement, or they might be kept proprietary, obscuring their details from end users and competitors. Protocol reverse engineering is most often employed in the latter scenario, where understanding the communication mechanisms is critical but official documentation is unavailable. The primary objective of protocol reverse engineering is to derive a model of how the target protocol functions, without any prior knowledge of its inner workings.

Protocol reverse engineering also plays a vital role in simulating network protocols. Network simulation allows researchers and developers to rapidly prototype specific test cases that might be cumbersome, time-consuming, or even impossible to execute in real-world environments. Furthermore, simulation tools can replay captured network traces under different conditions, helping engineers adapt their systems to new requirements or environments. In cybersecurity, protocol reverse engineering is frequently used in software security audits to verify that components behave correctly and securely under a range of circumstances, identifying potential vulnerabilities or compliance issues before they can be exploited.

Another significant application of protocol reverse engineering is network conformance testing, where the protocol in question is already known. In these cases, reverse engineering can help create a detailed model of the protocol’s behavior, which can then be compared to official standards or specifications to ensure that an implementation adheres to the expected rules and practices.

There are two general categories of protocol reverse engineering techniques. The first focuses on analyzing the messages exchanged between two components on a network. By carefully examining these exchanges—often referred to as network traces—analysts can infer the structure, commands, and responses that define the protocol’s behavior. The second approach involves reverse engineering the application itself. This can mean analyzing the source code (when available), disassembling binary code, or tracing sequences of binary instructions during execution to reveal how the protocol logic is implemented internally~\cite{stateofartRE}.

Many tools exist to automate significant parts of the protocol reverse engineering process. The initial setup phase typically involves identifying and characterizing the operating environment in which the protocol operates. With a solid understanding of the environment, analysts can begin the observation phase, which involves using specialized tools to collect network traffic or execution traces. Once data has been gathered, it must be sanitized to extract only the relevant messages associated with the target protocol, filtering out noise and unrelated data. The final step is the actual analysis, where analysts draw inferences about the protocol’s structure, message formats, and state machines. These steps are highly iterative and rely heavily on the expertise and insight of the analysts, as well as the capabilities of the tools being employed. Figure \ref{fig:protocolanalysis} illustrates this step-by-step process.
\begin{figure}[H]
	\caption{Protocol Analysis Process}
	\includegraphics[width=\linewidth]{protocol_analysis_process.png}
    \label{fig:protocolanalysis}
\end{figure}
\section{Compiled Code RE}
It is important for a reverser to understand the different goals of high level and low level languages. 
Understanding these differences will help with interpretation of why code is structured a certain way.

\subsection{High level languages}
High level languages exist as an abstraction to interface with operating system software and underlying hardware.
High level languages work so a programmer can focus on specifying program implementation logic and structural design without needing to dedicate time to figuring out system specific details. If a programmer were to decide to write instructions to the machine they would use assembly language, a low-level, human readable language used to communicate more directly with the computer hardware. Those instructions would then get turned into machine code, an even lower level programming language that is readable to a computer's processor.
Compiled languages are very performant as they are converted to assembly language then machine code. Writing directly in assembly language could allow for optimizations, but compilers are pretty good at enacting those optimizations themselves. When interpreters or VMs (like the Java Virtual Machine) are involved, the VM invokes the necessary behaviors unlike in languages such as C or C++ where the compiled code becomes machine code for the target platform to run directly. 
While writing in something like assembly language can be very performant, it would be significantly more difficult and time consuming for a person to write a modern application in assembly.
High level languages remove the need for programmers to understand their program at the assembly level, a reverse engineer however is forced to examine the low-level assembly version of a compiled program to understand the specific ways a program's goals are being achieved.
The main goal of a reverse engineer is to use what they know about a program's behavior, the potential ways that can be accomplished in high level code, and how to read assembly code to determine specific instrumentation details.
For reverse engineers, the most important thing to know about a high level language is to what level does it abstract or conceal the underlying machine code~\cite{Reversing}. 

Languages like Python that have a built in interpreter will abstract it a lot. 
These programs may be full of extraneous instructions to perform necessary operations. On the other hand, languages like C are written much closer to the target processor and won’t have nearly as much separation between the source and machine code

\subsubsection{Control Flow}
Control flow is what makes code determine what instructions get executed and how often.
This manifests through statements like conditionals that give general instructions for what a program should do when a condition is true or false.
A processor has no knowledge what statements like ‘if’ or ‘while’ mean. 
Under the hood, these statements translate into verbose and daunting assembly code. 
This is because high level conditional statements are often broken down into a series of smaller operation sequences to match the instruction set the processor understnads~\cite{Reversing}.

Other typical structural components of programs that aid in control flow are switch blocks and loops. 
Switch blocks, or \(n-way\) conditionals, take in an input and have \(n\) number of code blocks that are potentially executed depending on the input value. 
Each block of code gets assigned at least one value prior to runtime. 
The compiler also generates code to receive the input value and search for the proper code block to execute. 
The values for the code blocks are usually stored in a lookup table that has pointers to each corresponding code block.  
Depending on the input value, the program will go through the process of searching the lookup table then jumping to the proper code block at runtime~\cite{Reversing}. 
Loops work to allow a program to repeatedly execute a certain code block multiple times. A loop has a counter to keep track of how many iterations it has performed. 
There is also a conditional statement that determines when the loop will stop. 
Loops and conditionals are inherently intertwined. 
The difference is that loops execute over and over until the condition is no longer met.

To understand control flow sequences, one must understand how low level control flow is implemented. 
This means that a reverse engineer needs to know the specific rules of each kind of low level architecture because low level control flow instructions are unique to the platform for which they are designed.

\subsection{Low level languages}
Earlier we mentioned that complexity is introduced when working with low level system specific details. 
This is especially true when the goal is to translate high level logic into something that will be understood by your machine. 
Of course, there must be a way to do this because the CPU does it anytime a modern application is run. 
But it often involves keeping track of more details than a sole human is capable of. 
That is why a reverse engineer must develop the skill of parsing through assembly code and being able to create a kind of ‘mental image’ of how it relates to high level constructs.
Data management is one of the things that a person’s computer keeps track of that would be difficult for a human to do.
To understand why this is, we can look at the data management of a relatively low level language, C, versus the assembly representation. 
Consider the code: 
\begin{figure}[H]
\centering
\begin{minipage}{0.33\textwidth}
\caption{Divide function code}
\begin{lstlisting}
int divide(int a, int b) {
    int result;
    result = a / b;
    return result;
}
\end{lstlisting}
\end{minipage}%
\hfill
\begin{minipage}{0.3\textwidth}
    \centering
    \includegraphics[width=\linewidth]{divide_ghidra.jpeg}
    \caption{Disassembly view in Ghidra}
    \label{fig:ghidra1}
\end{minipage}%
\hfill
\begin{minipage}{0.3\textwidth}
    \centering
    \includegraphics[width=\linewidth]{divide_ghidra2.jpeg}
    \caption{Decompiled code view in Ghidra}
    \label{fig:ghidra2}
\end{minipage}
\end{figure}

While this function may seem incredibly simple, there is no direct translation into a machine code representation. 
To execute this in a low level language, it would require first storing the machine state. 
Then memory would need to get allocated for result. Variables a and b would need to get loaded from memory into a register. 
Then a would need to be divided by b and the result would need to get stored in the register that got loaded in the beginning. 
The machine state from before would need to get reloaded. The pointer would need to return to the caller and bring back result~\cite{Reversing}.
One line of high level code could result in any number of assembly instructions.
Managing data is one of the biggest challenges of reverse engineering.


\subsubsection{Registers}
To keep from needing to constantly access RAM, microprocessors have a smaller internal memory that can be accessed with barely any performance cost~\cite{Reversing}. 
There are multiple types of internal memories that microprocessors keep and registers are one of them.
Registers are little pieces of internal memory that live within the processor and are able to be accessed with an imperceptible cost to performance~\cite{PracticalRE}. Data must be in a register in order for the CPU to access it for processing. 
The biggest problem with registers is that there are not very many of them. 
One of the most popular processors, the IA-32, only has eight 32-bit registers that can be used for anything. 
It does have more registers, but they all have highly specific use cases. 
Assembly language is written around utilizing registers because they’re so performant. 
They are not good for long term storage, though, that is when RAM becomes the better choice.

RAM is volatile memory, meaning its contents are lost when the computer is powered off. It temporarily holds data and instructions for currently active tasks, giving the CPU quick access to the broader context of running programs. The CPU operates using something called the instruction cycle: first, it fetches an instruction; then, it decodes or interprets it to determine the required operation; finally, it executes the instruction, carrying out the specified task.

One of the most important things to take away is that CPUs do not automatically manage these tasks. 
Data management is outlined in the assembly code which is what reverse engineers will need to get comfortable sorting through.
One thing that reverse engineers will need to do is focus on figuring out what kind of values are getting loaded into a register. 
For example, it is easy to see when a register is only being used to grant instructions access to a specific value because that register will only show up when transferring value from memory to the instruction or vice versa.
Another example is when a register shows up many times in one function. 
This is a good clue that one of the function's local variables is being stored in that register. Figure \ref{fig:registerdiagram} shows some of the different registers and their purposes.
\begin{figure}[H]
	\caption{Register and Flag descriptions and purposes}
	\includegraphics[scale=.27]{Register Diagram.jpeg}
    \label{fig:registerdiagram}
\end{figure}
\footnote{Figure modified using Intel® 64 and IA-32 Architectures Software Developer’s Manual and https://flint.cs.yale.edu/cs422/doc/pc-arch.html\#register}

\subsubsection{The Stack}
The stack is one of two places that a value can be set aside. 
Registers are not managed by a processor and to use one you just need to load a value in it. 
Often there will not be registers available or there is a particular reason a variable will need to reside in RAM instead of in a register. 
That’s when you’d put a variable on the stack instead~\cite{Reversing}.

The stack is a short term storage space in memory that gets utilized by both the CPU and the program. 
It is the spot where short term information gets put when a register can not be used for whatever reason. 
Registers are for the shortest term data while the stack is for the second shortest. 
The stack lives in RAM like all other data and is just a carved out section for intermediate term data. 
Modern operating systems tend to manage multiple stacks at the same time. 
Each of the concurrently running stacks is a representation of a program or thread~\cite{MasteringRE}.

Stacks use Last In First Out data management. 
Items are pushed on the top of the stack and popped from the top of the stack. 
Memory in stacks is allocated from the top down where the first in address is allocated and used first while the stack grows backwards towards lower addresses~\cite{Reversing}.

\subsubsection{Flags}
One of the registers you may have noticed from the previous figure \ref{fig:registerdiagram} is the EFLAGS register. 
This is a collection of special IA-32 registers that contain system and status flags. 
The system flags’ job is to manage the different modes and states of the processor. 
The status flags are what a reverse engineer will typically be more interested in and are used by the processor to record its current logical state~\cite{Reversing}. 
They are often updated by various logical and integer instructions so they can record the outcome of these actions. 
There are also instructions whose operation conditions are dependent on the values for the status flags. 
This is what allows sequences of instructions to run different operations depending on what different input values may be, etc. 

Flags in IA-32 are the crux of conditional code. 
Arithmetic instructions check operand conditions and then set processor flag conditions based on the resulting values. 
There are also sets of instructions designed to read the flags and do different operations based on the value. 
An example of a popular instruction set is Jcc or the Conditional Jump. 
It tests for predefined flag values and then jump depending on whether or not it matches with the specified conditional code~\cite{intelManual}.

\subsubsection{Functions}
Instructions are the actual actions specified in assembly code. 
They are formatted with operation code (opcode) and one or two operands~\cite{Reversing} 
The opcode is what you are asking the computer to do such as MOV or JMP. 
The operand is the parameters that get passed to the opcode or the data being manipulated. 
Certain instructions will have no parameters. Data in assembly comes in three basic forms. 
There are register names, immediate data, and memory addresses. 
Register names are the names of the general purpose registers to be read from or written to like the EAX, EBX, etc. 
Immediate data is constant values that are embedded into the code and usually indicates there was a hard coded value in the source code. 
Memory addresses are the locations of operands stored in RAM. 
It can be hard coded and tell the processor where to read to and write from. 
It could also be a register with a value that will be used as a memory address. 
It is also possible to combine a register with an arithmetic operation and a constant so that there is some base address and then an index offset~\cite{Reversing}.

General purpose instructions are the ones that deal with program flow, logic, arithmetic, string operations, and basic data movement. 
They deal with data stored in memory at the general purpose registers, EFLAGS register, segment registers, and address information stored in memory ~\cite{Reversing}.

The MOV instruction shows up most frequently in most IA-32 instruction sets. 
This one deals with basic data movement. 
It takes in a destination operand and a source operand then moves the data from the source to the destination. 
The sources can be registers, immediate, or memory addresses. 
It is important to note that MOV cannot transfer data through memory, it can only take it out or put it in. 
The destination address can be a memory address (using a register or an immediate) or a register~\cite{Reversing}.




\section{Example Problems}
With some grounding in reverse engineering tools, methods, and assembly, it's a good time to try out a few practice problems. There are a couple of ways to approach this. One option is to work on “CrackMe” challenges, which are small programs designed to test reverse engineering skills by making analysis deliberately difficult. Another approach is to reverse a self-written program, placing markers in the code to check whether they can be identified during analysis. 

\subsection{Reversing Hello World}
Before taking on the larger endeavor of reversing a program written in a declarative language (SwiftUI), it’s important to start with the basics. 
Figure \ref{fig:helloworldsource} shows a simple ‘Hello World’ program in C. 
To add some complexity, a few random variables and data structures were thrown in as Easter eggs. 
This is the program whose decompilation will be used as a jumping off point.
\begin{figure}[h]
	\caption{Hello World Program}
	\includegraphics[scale=.75]{HelloWorldSource.png}
        \label{fig:helloworldsource}
\end{figure}
Figure \ref{fig:hellowworldghidra} is the decompilation of the program once it was put through Ghidra. 
The first thing to note is how with even a relatively low level language like C, the instructions in the code more than doubled. 
The disassembly being shown in the figure is also only a portion of Ghidra’s output. 
The figure only shows the function section of the program output when imports, exports, labels, classes, etc take up the large majority of the program. 
One important part of reverse engineering is sorting through all of the information to find out what exactly is important to the reverser. 
Some reverse engineers refer to this process as finding the shape and edges of the data. 
\begin{figure}[h]
	\caption{Hello World Disassembly}
	\includegraphics[scale=.3]{HelloWorldGhidra.png}
        \label{fig:hellowworldghidra}
\end{figure}
With the important section identified, it is time to analyze the assembly code.
One thing to keep in mind with the disassembled code is that all the values will be stored in hexadecimal. 
A good place to start is with the values we know will need to be stored. 
In the array there are 10 variables which all get set to 0, except for the eighth array member which is changed to equal 1. 
In our local\_dope struct there are three variables equal to ‘A’, ‘0xffffffff’, and 32000. 
The highlighted line in the figure shows ‘A’ being stored in the register w8. Shortly after, \#0xfffffff shows up without needing to be translated into hexadecimal. 
\#0x7d000 is the hex for 32000. 
This surface level analysis offers a glimpse into the methodology of more complex reverse engineering. 
Lastly, the 'Hello\_World!' gives us the final piece of information we need to know about what the program is.

\subsection{Crack Me}
The next step in practicing reverse engineering is trying a Crack Me problem. 
Crack Me problems are toy problems designed by other software developers to help reverser's practice their skills. 
Usually there’s an executable that opens up a little puzzle. 
The puzzles usually involve finding a password but can involve any number of things like decryption, key generation, etc. 
This section will cover the process of reverse engineering a simple crack me.

The first step I took in reversing this program was checking the header and libraries as shown in Figure \ref{fig:crackmeheader} and \ref{fig:crackmelibs}.
\begin{figure}[H]
	\caption{Crack Me Header}
	\includegraphics[width=\linewidth] {crackmeheader.png}
        \label{fig:crackmeheader}
\end{figure}
\begin{figure}[H]
    \centering
    \includegraphics[width=\linewidth]{crackmelibs.png}
    \caption{Crack Me Libraries}
    \label{fig:crackmelibs}
\end{figure}



Both of these steps were done with an executable dumping tool called ‘otool’. 
The header information shows that the assembly language being used is x86. 
The library dump shows that the only library being linked is the system library. 
While these tools didn’t offer that much information about this particular program, they are very useful in programs that have more dependencies and more specific assembly languages. 
These dumping tools ended up being vital later in the project because they offered insight into what libraries were being linked from the project's declarative language as well as what assembly language was being used with Apple’s modified version of ARM64. 
Next, otool was used once again to list the strings. 
This is a vital part of reverse engineering and is useful to be able to refer back to during the process. 
Figure \ref{fig:crackmestrings} shows the result of the string dump.


\begin{figure}[H]
	\caption{Crack Me Strings}
	\includegraphics[width=\linewidth]{crackmestrings.png}
        \label{fig:crackmestrings}
\end{figure}

Examining this more closely, you can see that there’s only a few strings that give a very clear idea of what the program is designed to do. 
There’s a random string, the title of the program, a passphrase prompt, a string indicating the expected input, a congratulations message, and an incorrect guess message. 
With just this information, you could easily make an educated guess on what the password is. 
Most likely the passphrase will be the one random string that doesn’t get given easily in the executable. 
But for the sake of gaining understanding of the process, it’s good to go over the decompiled code. 
Figure \ref{fig:crackmeghidra} shows the decompiled code.
\begin{figure}[H]
	\caption{Crack Me Ghidra Disassembly}
	\includegraphics[scale=.25]{crackmeghidra.png}
        \label{fig:crackmeghidra}
\end{figure}

The first thing to do is find the entrance into the program. 
Ghidra already labeled the ‘\_main’ at the top but the strings and function calls reaffirm this as the right section to be looking over. 
Next, thinking back to the string dump, it’s likely that the password will be near whatever function takes input. 
Prior knowledge of programming informs the reverser that after getting the user input, the input string will need to be compared to the right answer. 
Thankfully Ghidra was able to identify the function ‘scanf’.

\begin{figure}[H]
	\caption{Crack Me Code Section 1}
	\includegraphics[scale=.3]{crackmescanf.png}
        \label{fig:crackme1}
\end{figure}

Inspecting the code section at Figure \ref{fig:crackme1}, it shows that the variable local\_88 contains the user input. 
This is because it loads a \%99s into the RDI register to prepare for the input directly above where scanf is called. 
RSI has local\_1d loaded into it and that variable is immediately ‘strcmp’ compared to local\_88. Using these clues, one could infer that local\_1d contains the password. 
To verify, a reverser should always retrace and check.

\begin{figure}[H]
	\caption{Crack Me Code Section 2}
	\includegraphics[scale=.3]{crackmeRAX.png}
\end{figure}

The first instance of local\_1d contains the register RAX 9. 
RAX has the qword ptr ‘ s\_NoxIsTheBest\_100003f26’ moved into it right above. 
s\_NoxIsTheBest\_100003f25 only contains the string “NoxIsTheBest”. 
That string is probably the password, but all that’s left is to check. 
Figure \ref{fig:crackmeexe} shows the result.
\begin{figure}[H]
	\caption{Crack Me Code Executable}
	\includegraphics[scale=.6]{crackmeexe.png}
        \label{fig:crackmeexe}
\end{figure}
The attempt was a success and the crack me was successfully reverse engineered!


%!TEX root = ../main.tex
\chapter{Bluetooth}\label{text}
\section{Bluetooth}
Since this project involves reverse engineering a proprietary Bluetooth device, a foundational understanding of how Bluetooth operates is critical for decoding communication patterns and designing compatible interactions. Bluetooth is a widely adopted wireless technology that has seen rapid growth over the past decade. Originally developed to eliminate the need for short-range wired connections, Bluetooth is now used in countless applications—from music and video streaming to medical devices requiring continuous data updates. 
While it operates similarly to WiFi in some respects, the key difference is its scope: WiFi connects multiple devices to the internet, whereas Bluetooth focuses on direct device-to-device communication and offers more diverse use cases \cite{intelBLEguide}. Phones, headphones, cars, watches, keyboards, and countless other devices can all make use of Bluetooth. Any hardware that needs to interface with other devices can be designed to do so via Bluetooth.

\begin{figure}[h]
    \centering
    \includegraphics[scale=.5]{bluetoothdevices.png}
    \caption{Bluetooth Devices}
    \label{fig:bluetoothdevices}
\end{figure}
\footnotetext{Source: \url{https://www.klika.us/integrating-multiple-bluetooth-devices-into-a-network/}}
    
Another distinction is how Bluetooth is optimized for low-latency applications that transmit small bursts of data quickly and efficiently.
This makes it ideal for scenarios where power conservation and responsiveness are more important than high throughput.

\subsection{Bluetooth Types}

There are two main types of Bluetooth technology: \textit{Bluetooth Basic Rate (BR)}, also referred to as \textit{Bluetooth Classic}, and \textit{Bluetooth Low Energy (BLE)}. Of the two, BLE has gained far more traction in modern applications.

Bluetooth Basic Rate is the older of the two technologies, supported in versions 1.0 through 3.0. It is primarily used for wireless audio streaming in devices like headphones, speakers, and in-car entertainment systems, and it relies on point-to-point communication.

Bluetooth Low Energy, introduced in version 4.0 and further enhanced in version 5.0, offers a broader range of capabilities. In addition to supporting audio and data transfer, BLE enables device networking and location services. It supports multiple communication topologies including point-to-point, mesh, and broadcast, making it far more versatile than its predecessor. One of BLE’s standout features is its utility in high-accuracy location services. By leveraging nearby devices, BLE can determine relative positioning and provide precise location tracking.

\section{Pairing-Based Network}

The core of Bluetooth is creating a network where two (or slightly more) devices pair with each other, since Bluetooth was designed to replace corded connections. The devices being paired can be broken down into the \textit{Central} and \textit{Peripheral(s)}. These terms are specific to newer Bluetooth standards and can be used interchangeably with the older terms \textit{Master} and \textit{Slave}. In this section, I will use the newer terms.

The Central device coordinates the data being sent between it and any Peripherals. This includes handling time synchronization, sleep scheduling, and configuring the channels used through frequency hopping \cite{nextgenBLE}. Peripheral devices can communicate bi-directionally with the Central device, sending or receiving data depending on the request.

Figure \ref{fig:centralperipheral} shows an example of Central and Peripheral devices. Classic Bluetooth includes two network topology configurations: \textbf{Piconet} and \textbf{Scatternet}.

\begin{figure}[h]
    \caption{Central and Peripheral device relationship}
    \includegraphics[scale=.4]{centralperipheral.png}
    \label{fig:centralperipheral}
    \end{figure}

\subsection{Piconet}

A piconet is a type of ad hoc network topology that operates without preestablished infrastructure. Devices in a piconet communicate directly with one another without the need for a centralized device like a router or access point. The network is organized with one central device and one or more peripheral devices. A single central can support up to seven active peripherals, while each peripheral can only connect to one central \cite{nextgenBLE}. The central manages the timing and synchronization of all devices in the piconet. Peripheral devices cannot communicate directly with each other; the central acts as a relay, forwarding data between them.

Figure \ref{fig:piconet} shows the structure of a piconet.

\begin{figure}[h]
    \caption{Piconet Topology}
    \includegraphics[scale=.6]{piconet.png}
    \label{fig:piconet}
    \end{figure}

All connections made using a BR/EDR controller occur within a piconet. BR/EDR devices communicate over the same physical channel by synchronizing with a shared clock and hopping sequence \cite{bluetoothcorespec6}. The piconet clock follows the central device's clock configuration, and the hopping sequence is also derived from the central’s clock and Bluetooth device address.

Multiple independent piconets can exist in the same area. Each operates on a separate channel and is organized around a different central device. Bluetooth devices can also participate in multiple piconets simultaneously. While a Bluetooth device can only act as the central in one piconet at a time, it can serve as a peripheral in several.


\subsection{Scatternet}

A scatternet is formed by connecting multiple piconets together. Unlike a piconet, a peripheral in a scatternet can connect to multiple central devices. Central devices can also connect to each other, sometimes serving dual roles as both central and peripheral.

For example, a scatternet might include a central laptop connected to a Bluetooth speaker, while also acting as a peripheral to a smartphone and printer. Figure \ref{fig:scatternet} shows a diagram of a typical scatternet. Despite the increased complexity, peripherals still do not communicate directly with each other.

Piconets and scatternets are associated with Classic Bluetooth but are still supported in BLE and later Bluetooth versions \cite{nextgenBLE}. Although BLE—with its advertising mode—has seen growing popularity, traditional Bluetooth methods still offer advantages such as added security during pairing. BLE services enhance the piconet and scatternet model, providing better security, more efficient pairing, and broader support for services.

\begin{figure}[h]
    \caption{Scatteret Topology}
    \includegraphics[scale=.7]{scatternet.png}
    \label{fig:scatternet}
    \end{figure}

\section{BLE Mesh Networks}

Mesh networking, introduced with BLE, marks a significant evolution in Bluetooth topologies. In a mesh network, nodes are arranged in a self-configuring mesh, where each node relays traffic between peers when the destination is not directly connected \cite{buildingBLEsystems}.

This structure ensures a reliable path for data packets. If one path is blocked or unavailable, the network reroutes traffic dynamically. The self-organizing nature of mesh networks means they adapt in real-time to changes, such as nodes joining, leaving, or moving within the network.

For example, Figure \ref{fig:mesh} shows Device 1 out of range of Device 3. Device 2, within range of both, acts as a router forwarding messages between them.
\begin{figure}[H]
    \caption{Mesh topology}
    \includegraphics[scale=.8]{mesh.png}
    \label{fig:mesh}
    \end{figure}

\section{Broadcast Network from LE}

BLE introduces a new feature called \textit{advertising mode}, which operates differently from the traditional pairing between a central and peripheral. In advertising mode, a device broadcasts data openly to any BLE devices within range that are tuned to the same channel. These listening devices can receive the data and use it as needed—without any pairing, connection, or formal association.

This topology loosely resembles a multi-peripheral piconet, but with a key difference: advertising mode removes the need for a dedicated central or established links. There’s no synchronization or connection—just open broadcasting.

Advertising mode has been a major factor in expanding the possibilities of Bluetooth. By offloading the overhead of pairing, BLE made room for a wider variety of lightweight, low-latency applications. This performance shift is part of why so many modern devices—from smart beacons to virtual reality headsets—are feasible today. Devices like VR headsets, which require fast, efficient data sharing without constant handshaking, would be nearly impossible to build the same way without Bluetooth.

\section{Connections and Mixed Topology}

A connection becomes necessary when data need to flow in both directions or when there’s more data than can fit into the two available advertising payloads. Connections in BLE are more persistent—they allow devices to remember each other and avoid reestablishing the connection every time. Once established, connections support periodic data exchanges between two devices.

This connection-based pairing is what underpins piconets and scatternets. These connections are private by design: only the two participating devices exchange data (barring external sniffing). As covered earlier, connections involve one central device and one or more peripheral devices.

BLE networks can also incorporate mixed topology configurations depending on the use case. A device that supports both BR/EDR and LE can serve as a bridge between traditional Bluetooth and BLE communication, enabling broader interoperability across protocols. The possibilities for combining these configurations are only constrained by the capabilities of the individual devices' radios and protocol stacks.

Figure \ref{fig:mixedtopology} provides an example of a mixed topology Bluetooth network.

\begin{figure}[H]
    \caption{Mixed Topology}
    \includegraphics[scale=.8]{mixedtopology.png}
    \label{fig:mixedtopology}
    \end{figure}

\section{SCO and ACL Logical Transports}
With an understanding of how Bluetooth device networks are organized, we can now narrow our focus to how data is actually communicated within these networks. This communication process is crucial to keep in mind when reverse engineering a Bluetooth device.

\subsection{BR/EDR Asynchronous Connection-Oriented (ACL)}

ACL links are the primary way Bluetooth devices using BR/EDR exchange general data. These links handle both control signals (like LMP and L2CAP) and typical user data. ACL is \textit{asynchronous}, meaning data is sent when available rather than on a fixed schedule.

Every Peripheral in a Bluetooth piconet gets one default ACL link to the Central. This link is established when the device first connects and is identified by a Logical Transport Address (LT\_ADDR) assigned by the Central. This LT\_ADDR is also used when identifying the physical connection between devices.

However, because multiple types of logical transports (like SCO links, which are used for voice) might use the same LT\_ADDR, it’s not enough to identify the ACL connection by LT\_ADDR alone. Devices also rely on the packet type to tell them which kind of transport is being used.

ACL links can also carry isochronous data—data that needs to arrive on time, like audio streams—by setting them to automatically drop old packets. At the same time, asynchronous traffic (like file transfers) can still be sent if it’s marked not to auto-flush. This means both types of traffic can share the same ACL link if configured correctly.

If the ACL link is removed or if the device loses sync with the piconet, all other connections between the Central and that Peripheral (like SCO) are also dropped immediately \cite{bluetoothcorespec6}.

\subsection{BR/EDR Synchronous Connection-Oriented (SCO)}

SCO links are designed specifically for real-time data like voice. They create a fixed, circuit-like connection between a Central and a Peripheral by reserving slots on the Bluetooth physical channel. These connections carry 64 kbps of data, usually for audio, and are synchronized with the piconet’s clock.

There are several SCO configurations that balance audio quality, latency, and bandwidth usage. Every SCO connection uses the same LT\_ADDR as the ACL link, so to identify an SCO packet, a device must look at the slot number, packet type, and LT\_ADDR together.

Even though SCO reserves bandwidth, the system can override these slots if higher-priority data (like control messages over ACL) needs to get through. This helps maintain overall system reliability and meet Quality of Service (QoS) requirements.

Unlike ACL links, SCO links don’t carry complex protocols or layered data structures. They just send a steady stream of audio—either in a standard format or as raw data for the application to decode \cite{bluetoothcorespec6}.

\section{Bluetooth Low Energy Broadcasting and Observing}

Bluetooth Low Energy devices communicate with the outside world in two primary ways: \textit{broadcasting} and \textit{connecting}. Both methods are governed by the Generic Access Profile (GAP), which outlines how devices discover and interact with each other \cite{gettingstartedwble}.

Broadcasting enables BLE devices to send data one-way to any scanner or receiver within range. It’s a form of open communication where any nearby device capable of listening can receive the broadcasted data. Figure \ref{fig:broadcastobserve} illustrates this communication model. In the diagram, two roles are shown: the \textit{Broadcaster}, which periodically sends non-connectable advertising packets, and the \textit{Observer}, which continually scans for those packets \cite{gettingstartedwble}.

\begin{figure}[H]
    \caption{Broadcasting and Observing}
    \includegraphics[scale=.8]{broadcastobserve.png}
    \label{fig:broadcastobserve}
    \end{figure}

Broadcasting is a key concept in BLE because it’s the only mechanism that allows a device to communicate with multiple other devices simultaneously. This functionality is made possible by BLE’s advertising feature.

A standard advertising packet contains a 31-byte payload, which includes information about the broadcasting device and what it offers. This data can be customized to suit different applications, providing useful context to observing devices. If 31 bytes isn’t enough, BLE allows for a secondary packet called a \textit{Scan Response}, which provides an additional 31 bytes of payload. Observers can request this secondary data if they need more information \cite{gettingstartedwble}.

Broadcasting is quick, efficient, and ideal for sending small amounts of data on a regular schedule to multiple receivers. However, it does come with limitations—particularly around security. Since any device within range can intercept a broadcast, there’s virtually no privacy. Because of this, broadcasting isn’t well-suited for transmitting sensitive or private data.

\section{Bluetooth Protocols and Profiles}
Bluetooth technology is organized into two key structural layers: protocols and profiles. Protocols form the lower-level foundation used by all Bluetooth-enabled devices. These protocols handle the encoding, decoding, packet formatting, multiplexing, and routing required for reliable wireless communication \cite{gettingstartedwble}.
Profiles, on the other hand, are higher-level constructs that define how these underlying protocols should be used in practice. Often described as “vertical slices” of functionality, profiles ensure interoperability between devices by prescribing behavior for specific operations or applications. Profiles are typically divided into two categories: generic profiles, such as the GAP and Generic Attribute Profile (GATT), which are essential to all Bluetooth Low Energy (BLE) communications, and use-case-specific profiles, like the Proximity or Glucose Profile, which layer additional functionality on top of the generic ones to meet specialized needs.

\subsection{Generic Profiles}

Generic profiles are defined by the Bluetooth Core Specification and provide the foundational mechanisms required for BLE communication. These profiles ensure that devices from different manufacturers can interoperate seamlessly. Two such profiles, GAP and GATT, are mandatory for all BLE-compliant devices and serve as the cornerstones of BLE communication.

\subsubsection{Generic Access Profile (GAP)}

The Generic Access Profile defines the roles, procedures, and operational modes required for device discovery, broadcasting, connection establishment, connection management, and security negotiation. GAP operates as the top-level control layer within the BLE protocol stack, orchestrating the basic behaviors necessary for device interaction. All BLE devices must implement and conform to the GAP specifications to ensure compatibility and reliable communication.

\subsubsection{Generic Attribute Profile (GATT)}

GATT focuses primarily on how data is exchanged. GATT defines a universal data model and a set of procedures that enables devices to discover, read, write, and notify data values. It functions as the uppermost data layer in BLE and serves as the foundational framework upon which most BLE applications and services are constructed.

Due to their foundational roles, GAP and GATT are commonly exposed through application programming interfaces (APIs), making them the primary entry points for developers interacting with the BLE protocol stack.

\subsection{Use-Case-Specific Profiles}

Beyond the generic profiles, the Bluetooth Special Interest Group (SIG) has defined a series of use-case-specific profiles. These profiles are designed to standardize behavior for specific applications and are exclusively built upon the GATT framework. At the time of writing, no BLE profiles exist outside the GATT structure. However, the introduction of connection-oriented L2CAP channels in Bluetooth version 4.1 may pave the way for future GATT-less profiles.

\subsubsection{SIG-Defined GATT-Based Profiles}

The Bluetooth SIG provides an extensive catalog of standardized profiles, each tailored to support a specific application or device type. These profiles fully specify the required procedures and data formats, facilitating rapid development and ensuring interoperability across implementations.

There are several GATT-based profiles defined by the Bluetooth SIG that standardize common BLE use cases. The Find Me Profile helps locate nearby BLE devices, such as a lost phone or keyring. The Proximity Profile detects when a connected device moves beyond a certain range and can trigger alerts. The HID over GATT Profile is designed for peripherals like keyboards, mice, and remote controls, enabling the transmission of Human Interface Device (HID) data over BLE. The Glucose Profile supports the secure transmission of glucose measurement data for health monitoring. Similarly, the Health Thermometer Profile allows body temperature readings to be sent from wearable or medical devices. Lastly, the Cycling Speed and Cadence Profile enables cycling sensors to communicate speed and cadence information to compatible applications or devices.

A comprehensive list of SIG-adopted profiles is available on the Bluetooth SIG Specification Adopted Documents webpage. Developers may also consult the Bluetooth Developer Portal for up-to-date listings of adopted services and characteristics.

\subsubsection{Vendor-Specific Profiles}

While the SIG provides a broad range of standardized profiles, the BLE specification also permits the definition of vendor-specific profiles. These profiles are often developed to support proprietary use cases not covered by SIG standards. Vendor-specific profiles can be implemented privately between two devices (such as with a health accessory and a smartphone) or published to enable broader adoption.

Notable examples of vendor-defined profiles include profiles like Apple iBeacon which is a proprietary proximity-sensing profile used for indoor positioning and location-based services. Another example is the Apple Notification Center Service (ANCS) which enables wearable devices or external displays to access and display iOS notifications.

These profiles demonstrate the flexibility of the BLE specification in accommodating both standardized and proprietary communication models.

\section{Bluetooth Low Energy (BLE) Architecture}

Although most users will only ever interact with the top layers of the Bluetooth Low Energy (BLE) protocol stack, having a basic understanding of the full stack can provide helpful context. This overview lays the groundwork for understanding how BLE devices work and why each part of the system is important. A typical single-mode BLE device is made up of three main parts: the controller, the host, and the application. Each of these parts contains multiple layers, each responsible for specific tasks that allow the device to communicate and perform its intended functions. Figure \ref{fig:bleprotocolstack} shows the BLE protocol stack.

\begin{figure}[h]
    \caption{BLE Protocol Stack}
    \includegraphics[scale=.7]{bleprotocolstack.png}
    \label{fig:bleprotocolstack}
    \end{figure}

\subsection{Application}

The application layer sits at the top of the BLE protocol stack. It includes all the logic, data handling, and user interface components that define how a device behaves in a specific use case. This layer is highly customized and varies from one implementation to another \cite{nextgenBLE}.

\subsection{Host}

Below the application is the host, which includes several critical layers that manage communication and device behavior.

The Bluetooth Low Energy protocol stack includes several key components, each with a specific role. The \textbf{Generic Access Profile (GAP)} controls how devices advertise themselves, discover others, establish connections, and manage roles and security \cite{nextgenBLE}. The \textbf{Generic Attribute Profile (GATT)} is responsible for organizing and managing how data is exchanged between connected devices. The \textbf{Logical Link Control and Adaptation Protocol (L2CAP)} handles multiplexing and segmentation of data packets. The \textbf{Attribute Protocol (ATT)} enables data operations such as reads and writes using attribute handles. The \textbf{Security Manager (SM)} oversees pairing, authentication, and encryption processes. Lastly, the \textbf{Host Controller Interface (HCI) – Host Side} facilitates communication between the host and the controller \cite{nextgenBLE}.

\subsection{Controller}

At the bottom of the Bluetooth stack, the controller is responsible for handling radio operations and low-level data transport. It typically consists of several core components. The \textbf{Physical Layer (PHY)} manages the actual transmission and reception of radio signals. The \textbf{Link Layer (LL)} controls tasks such as advertising, scanning, and maintaining active connections. The \textbf{HCI (Controller Side)} works in conjunction with the host side of the Host Controller Interface to enable structured data flow between the different layers \cite{nextgenBLE}.

According to the Bluetooth Core Specification, a Bluetooth implementation includes a single controller, which can be configured in one of several ways. A \textbf{BR/EDR controller} contains the Radio, Baseband, Link Manager, and optionally the Host Controller Interface (HCI). An \textbf{LE controller} includes the LE PHY, Link Layer, and may also include the HCI. Alternatively, a \textbf{dual-mode controller} combines both BR/EDR and LE controller functionalities into a single integrated unit.

Together, these layers ensure that BLE devices can connect reliably and communicate efficiently. This layered architecture is often described from the bottom up — starting at the antenna and PHY layer and moving up through the stack to the user-facing application.

\section{BLE Link Layer and Device Addressing}

In the Bluetooth Low Energy (BLE) protocol stack, the Link Layer serves as a crucial interface between the Physical Layer—which handles the actual radio transmission—and the upper protocol layers. It acts as the controller for radio operations and is essential for managing the states and timing mechanisms that define BLE communication. One of the key responsibilities of the Link Layer is to abstract the radio hardware, allowing higher-level layers to interact with it through the Host Controller Interface (HCI) \cite{introtoble}.

In addition to timing and control, the Link Layer manages several hardware-accelerated tasks. These include generating cyclic redundancy checks (CRC) for error detection, producing random numbers for cryptographic operations, and performing encryption to secure data transfer \cite{introtoble}. These operations ensure the performance, integrity, and security of BLE communications.

\subsection{Link Layer States}

BLE devices operate in a well-defined set of five main states, each with specific behaviors and transitions:
\begin{itemize}
    \item \textbf{Standby:} This is the default state, where the radio is idle and not transmitting or receiving any packets.
    \item \textbf{Advertising:} In this state, the device broadcasts data packets at regular intervals, making itself discoverable to other devices.
    \item \textbf{Scanning:} Devices in this state listen for advertising packets from others, searching for devices they might want to connect with.
    \item \textbf{Initiating:} When a scanning device decides to connect to an advertising device, it enters this state and sends a connection request.
    \item \textbf{Connected:} A persistent communication link is established. Devices in this state can exchange data regularly. The device that initiated the connection becomes the master, while the other becomes the slave \cite{introtoble}.
\end{itemize}
\begin{figure}[H]
    \centering
    \includegraphics[width=0.45\linewidth]{linklayerstates.png}
    \caption{Link Layer State Machine}
    \label{fig:linklayerstates}
\end{figure}
This framework enables BLE to support dynamic and responsive device discovery and communication, while keeping energy consumption minimal.
\subsection{Advertising, Scanning, and Connecting}

Advertising and scanning are complementary operations: advertising devices periodically transmit small packets of data, while scanning devices listen for and interpret these packets. If the advertiser allows connections, and a scanning device decides to initiate one, both devices transition into the connected state \cite{introtoble}. These processes form the backbone of BLE interactions and are covered in more depth in subsequent chapters of most BLE technical literature.

\subsection{Bluetooth Device Address}

Every Bluetooth device is identified by a 48-bit device address, which serves a function similar to a MAC address in networking. BLE supports two major types of device addresses: public addresses and random addresses, which influence how discoverable and trackable a device is.
A public address is a globally unique, fixed identifier that is factory-programmed and registered with the IEEE. 

\begin{figure}[H]
    \centering
    \includegraphics[width=0.5\linewidth]{bluetoothaddress.png}
    \caption{Example Bluetooth Address}
    \label{fig:bluetoothaddress}
\end{figure}

Because of this registration requirement, it ensures uniqueness across all BLE devices worldwide. However, it can also present privacy concerns, since it does not change and is easily traceable \cite{introtoble}.
Random addresses, by contrast, provide more flexibility and are often preferred in consumer devices to protect user privacy. These addresses can either be static or private, and they do not require IEEE registration.

\textbf{Static Addresses} are used as a drop-in replacement for public addresses. A static address remains the same between sessions but can be regenerated at boot or kept for the lifetime of the device until a power cycle occurs.

\textbf{Private Addresses} are designed to change regularly to enhance privacy. There are two subtypes of private addresses, \textbf{Non-resolvable private addresses} and \textbf{Resolvable private addresses}. Non-resolvable private addresses are temporary, randomly generated, and cannot be traced back to a specific device. They are rarely used due to limited functionality. Resolvable private addresses use an Identity Resolving Key (IRK) and a random number to generate temporary addresses. These change over time, making the device hard to track by unknown parties. However, trusted or bonded devices that have the IRK stored can resolve these addresses, allowing them to re-identify the device securely \cite{introtoble}.

This system allows BLE to balance two key goals: persistent connectivity with trusted devices and protection against unwanted tracking or eavesdropping by unknown entities.

\section{Host Controller Interface (HCI) Layer}

The Host Controller Interface (HCI) serves as the communication bridge between the host and controller layers in a Bluetooth system. It is what enables communications between the lower layers of the bluetooth stack. These components can be implemented on the same chipset or on separate chipsets. When they are on separate chipsets, HCI is critical for enabling interoperability and communication between them. In such cases, HCI defines both the standardized protocol and the physical transport mechanisms that facilitate message exchange. As shown in Figure \ref{fig:hosthcicontroller}, the host, HCI, and controller are depicted as distinct components. HCI commonly uses transport technologies such as UART, USB, and SDIO to establish physical connections between the host and controller. When the host and controller are integrated on the same chipset, HCI functions as a logical interface rather than a physical one \cite{introtoble}.
\begin{figure}[H]
    \centering
    \caption{Host, HCI, and Controller components}
    \includegraphics[scale=.35]{hosthcicontroller.png}
    \label{fig:hosthcicontroller}
    \end{figure}

\subsection{Universal Asynchronous Rx/Tx (UART)}

UART is a serial communication protocol that uses wired connections to exchange data between devices. It is commonly found in integrated circuits and is especially favored in devices like mobile phones and laptops, where the entire Bluetooth stack is implemented on a single chipset. UART is also well-suited for small, resource-constrained devices powered by microcontrollers, as it is lightweight and efficient in its use of system resources \cite{nextgenBLE}.

In the context of the HCI UART protocol, four types of packet transmissions are defined: the Command packet (identified by 0x01), Event packet (0x04), Asynchronous Connection-Less (ACL) packet (0x02), and Synchronous Connection-Oriented (SCO) packet (0x03). Because the receiving HCI layer cannot distinguish these packet types based on content alone, each transmission begins with a leading byte — known as an indicator — which acts as a header to identify the type of packet that follows \cite{nextgenBLE}.

\section{Connection}

To establish a connection in Bluetooth Low Energy (BLE), the central device begins by scanning for advertising peripheral devices that are currently accepting connection requests. Advertising packets can be filtered based on the Bluetooth Address or the contents of the advertising data itself. Once a suitable advertiser is detected, the master sends a connection request packet. If the slave responds, a connection is established \cite{gettingstartedwble}.

The connection request packet includes several critical parameters that define how communication will proceed. One such parameter is the frequency hop increment, which determines the hopping sequence used throughout the connection. In addition, the packet specifies three connection-related parameters. The connection interval defines the time between the start of two consecutive connection events, ranging from as short as 7.5 milliseconds for high throughput to as long as 4 seconds for reduced power consumption. The slave latency indicates how many connection events the slave device is permitted to skip without risking disconnection. Lastly, the connection supervision timeout sets the maximum duration allowed between the receipt of valid packets before the connection is considered lost \cite{gettingstartedwble}.

Once connected, communication happens through repeated connection events, which are periodic time slots where the master and slave take turns exchanging packets. A connection event always begins with a packet from the central, requires the peripheral to respond if it receives a packet, continues until both sides have no more data to send, is repeated at the connection interval until the connection is either closed or lost, and can be closed by either the master or the slave. If the central sends a packet and does not receive a response, it waits until the next connection event to resume.
To manage interference and ensure privacy or security, BLE supports a white list mechanism at the Link Layer. This list defines which Bluetooth device addresses are of interest. Devices not on the list are ignored — their advertising (by scanners) or connection request packets (by advertisers) are simply dropped \cite{gettingstartedwble}.

\section{Services and Characteristics}
In Bluetooth Low Energy, services and characteristics define how data is structured, communicated, and managed on a server. A clear understanding of this architecture is essential for tracing data flows and identifying commands in any BLE-enabled device.

\subsection{Attribute Protocol (ATT)}

The Attribute Protocol (ATT) defines how a server exposes its data to a client, and how that data is organized and accessed in a Bluetooth Low Energy (BLE) system \cite{introtoble}.

\subsubsection{Roles in ATT}

There are two main roles in ATT communication:
\begin{itemize}
    \item \textbf{Server:} The server is the device that stores and exposes data, and may allow certain behaviors to be remotely controlled. It receives commands from peer devices and sends responses, notifications, or indications in return. For example, a thermometer acts as a server when it provides access to its current temperature, unit of measurement, battery level, or the interval at which it records readings. Instead of requiring the client to repeatedly poll for changes, the server can proactively notify the client when data updates occur \cite{introtoble}.
    \item \textbf{Client:} The client is the device that reads data from the server or controls the server's behavior. It sends commands and requests, and accepts incoming notifications and indications. In the thermometer example, a mobile device that connects to the thermometer and reads temperature values is operating as the client \cite{introtoble}.
\end{itemize}

\subsubsection{Data Format: Attributes}

The data exposed by the server is organized into attributes. An attribute is a general term for any piece of data available on the server, such as services or characteristics (which are described further in the text).

Each attribute in the Bluetooth Attribute Protocol (ATT) is composed of several key elements. The first is the attribute type, identified by a Universally Unique Identifier (UUID), which distinguishes the kind of data the attribute represents. A 16-bit UUID is used for standard attributes adopted by the Bluetooth SIG, while a 128-bit UUID is reserved for custom or vendor-specific attributes defined by developers \cite{introtoble}. The second element is the attribute handle, a 16-bit value assigned by the server that functions like an address. This handle uniquely identifies each attribute during the connection and allows the client to reference it directly. Handle values range from 0x0001 to 0xFFFF, with 0x0000 being reserved \cite{introtoble}. Lastly, attribute permissions determine what operations—such as read, write, notify, or indicate—are allowed on the attribute and what security conditions must be met. These permissions are defined outside of the ATT layer, typically in the GATT or in the application logic \cite{introtoble}.
\section{Profiles}

\subsection{Generic Attribute Profile (GATT)}

The GATT defines how BLE devices organize, expose, and exchange data over a connection. It builds directly on top of the Attribute Protocol (ATT), using it as the transport layer for transmitting structured data known as attributes \cite{introtoble}.

GATT introduces a hierarchical data model centered around three key concepts: Services, Characteristics, and Profiles \cite{introtoble}.

These elements allow BLE devices to encapsulate user-facing data and device functionality in a standardized, interoperable way. All BLE application-level interactions happen within the framework defined by GATT \cite{gettingstartedwble}.

\subsection{GATT Roles}

Like ATT, GATT supports client and server roles, which are dynamic per transaction rather than fixed per device. This means a single device can simultaneously act as a GATT server for one connection and a GATT client in another \cite{introtoble}.

\begin{itemize}
    \item \textbf{GATT Server:} Stores and exposes attributes to the client. It responds to requests and may send unsolicited updates via notifications or indications. Every BLE device must at least support a minimal GATT server, even if only to respond with error codes \cite{gettingstartedwble}.
    \item \textbf{GATT Client:} Initiates communication by performing service discovery, reading/writing attributes, and subscribing to updates. The client has no prior knowledge of the server’s structure—it learns through discovery procedures \cite{gettingstartedwble}.
\end{itemize}

\subsection{Services}

A service is a logical grouping of one or more attributes that collectively provide a certain function on the server. A service always includes at least one characteristic, and often contains supporting attributes like declarations, descriptors, and included services (used to reference other services) \cite{introtoble}.

There are two main classifications for services. There are \textbf{Primary Services} which represent core functionality for a Bluetooth device. Services such as Battery Service or Device Information Service are primary services. Primary services are more commonly used than the other type of service, \textbf{Secondary Services}. Secondary services offer auxilary support and must be referenced by primary services \cite{introtoble}.

The Bluetooth SIG maintains a set of adopted services with published specifications to ensure interoperability across vendors. If a product claims compliance with one of these services, it must strictly follow its specification \cite{introtoble}.

\subsection{Characteristics}

A characteristic is the smallest logical unit of user-accessible data on a BLE server. Characteristics always belong to a service. They also include values which contain the actual data being exposed, properties which are the operations permitted to be performed on the value (e.g., read, write, notify, indicate), and metadata about the value (e.g., format, units, configuration settings).

For example, the battery level characteristic within the Battery Service lets clients read the device's current power level \cite{introtoble}.

\subsection{Profiles}

While services and characteristics define how data is stored and exposed on the server, profiles describe how clients and servers interact, including service usage, connection procedures, and security requirements. Profiles are not discovered over BLE connections like services are; instead, they exist as specification documents, often adopted by the Bluetooth SIG. These define how multiple services should be used together to achieve specific use cases (e.g., Heart Rate Profile, Blood Pressure Profile) \cite{introtoble}.

Each profile specification includes: required services and characteristics, interaction behaviors for both server and client, connection and security requirements, and example usage diagrams and workflows \cite{introtoble}.


\section{Nordic Semiconductor}

\subsection{nRF Connect}

nRF Connect for Mobile—formerly known as the nRF Master Control Panel—is a feature-rich application designed for developers working with Bluetooth Low Energy (BLE) devices. Created by Nordic Semiconductor, this free app is available on both Android and iOS platforms and is widely regarded as one of the most powerful tools for BLE development and testing \cite{buildingBLEsystems}.

The application supports numerous Bluetooth SIG-adopted profiles, including Over-the-Air Device Firmware Updates (OTA DFU), making it especially useful for firmware deployment and debugging \cite{buildingBLEsystems}. Key features include BLE device scanning with real-time RSSI graph visualization, advertisement data parsing for both general and manufacturer-specific payloads, and comprehensive GATT client functionality that allows connection to peripherals and inspection of their services and characteristics. Users can interact with characteristics by performing read, write, and observe operations, and can enable notifications and indications to receive real-time updates from connected devices. The application also supports GATT server emulation with built-in presets, OTA firmware updates for supported devices, and automated testing through XML-defined test scripts to validate BLE device behavior. Additionally, it provides UART service support for communication over UART-over-BLE channels \cite{buildingBLEsystems}.

It is compatible with Android version 4.3 and higher or iOS version 8.0 and higher. At the time of writing, the Android version supports more functionality than the iPhone version.

With its comprehensive feature set, nRF Connect is an essential tool for BLE developers, from prototyping and debugging to final deployment \cite{buildingBLEsystems}.
\input{chapters/chapter3}
\input{chapters/chapter4}
\input{chapters/chapter5}
%\input{chapters/chapter6}
%\input{chapters/chapter7}
%\input{conclusion}

%%%%%%%%%%%%%%%%%%%%%%%%%%%%%%%%%%%%%%%%%%%%%%%%%%%%%%%
%
%  This section starts the back matter. The back matter includes appendices, indicies, and the
%  bibliography
%
%%%%%%%%%%%%%%%%%%%%%%%%%%%%%%%%%%%%%%%%%%%%%%%%%%%%%%%

\backmatter

%%%%%%%%%%%%%%%%%%%%%%%%%%%%%%%%%%%%%%%%%%%%%%%%%%%%%%%
%
%  We used BibLaTeX and Biber to generate a Bibliography.
%
%%%%%%%%%%%%%%%%%%%%%%%%%%%%%%%%%%%%%%%%%%%%%%%%%%%%%%%

%\nocite{*} % This command forces all the bibliography references to be printed -- not just 
              % those that were explicitly cited in the text.  If you comment this out, the bibliography
              % will only include cited references.
\printbibliography[title=References,heading=bibintoc]% load our Bibliography file

%%%%%%%%%%%%%%%%%%%%%%%%%%%%%%%%%%%%%%%%%%%%%%%%%%%%%%%
%
%                                                                Index
%
%  Uncomment the lines below to include an index. To get an index you must put 
%  \index{index text} after any words that you want to appear in the index.
%  Subentries are entered as \index{index text!subentry text}. You must also run the
%  makeindex program to generate the index files that LaTeX uses. The PCs are set to run
%  makeindex automatically.
%
%%%%%%%%%%%%%%%%%%%%%%%%%%%%%%%%%%%%%%%%%%%%%%%%%%%%%%%

\ifthenelse{\boolean{index}}{
\cleardoublepage
\phantomsection
\addcontentsline{toc}{chapter}{Index}
\printindex}{}

%%%%%%%%%%%%%%%%%%%%%%%%%%%%%%%%%%%%%%%%%%%%%%%%%%%%%%%
%
%                                                                Colophon
%
%  A Colophon is a section of a printed document that acknowledges the designers and printers of the work.
% The colophon also includes information about the fonts and paper used in the printing. It is not required 
% for your IS and can be commented out.
%
%%%%%%%%%%%%%%%%%%%%%%%%%%%%%%%%%%%%%%%%%%%%%%%%%%%%%%%

\ifthenelse{\boolean{colophon}}{
\begin{colophon}
This Independent Study was designed by Dr. Jon Breitenbucher.\newline
It was edited and set into type in Wooster, Ohio,\newline
using the \ifthenelse{\boolean{xetex}}{\XeTeX\ typesetting system designed by Jonathan Kew}{\LaTeX\ typesetting system designed by Leslie Lamport}\newline
and based on the original \TeX\ system of Donald Knuth.\newline

The text face is Palatino, designed by Hermann Zapf in 1948,\newline
and initially released in 1949 by the Stempel foundry and later\newline
by other companies, most notably the Mergenthaler Linotype\newline
Company. Palatino is optimised for legitibility with open counters,\newline
balanced proportions, moderate stroke contrast and flared serifs.
\end{colophon}}{}
\clearpage\thispagestyle{empty}\null\clearpage
\end{document}